%% Generated by Sphinx.
\def\sphinxdocclass{report}
\documentclass[letterpaper,10pt,english]{sphinxmanual}
\ifdefined\pdfpxdimen
   \let\sphinxpxdimen\pdfpxdimen\else\newdimen\sphinxpxdimen
\fi \sphinxpxdimen=.75bp\relax
\ifdefined\pdfimageresolution
    \pdfimageresolution= \numexpr \dimexpr1in\relax/\sphinxpxdimen\relax
\fi
%% let collapsible pdf bookmarks panel have high depth per default
\PassOptionsToPackage{bookmarksdepth=5}{hyperref}

\PassOptionsToPackage{warn}{textcomp}
\usepackage[utf8]{inputenc}
\ifdefined\DeclareUnicodeCharacter
% support both utf8 and utf8x syntaxes
  \ifdefined\DeclareUnicodeCharacterAsOptional
    \def\sphinxDUC#1{\DeclareUnicodeCharacter{"#1}}
  \else
    \let\sphinxDUC\DeclareUnicodeCharacter
  \fi
  \sphinxDUC{00A0}{\nobreakspace}
  \sphinxDUC{2500}{\sphinxunichar{2500}}
  \sphinxDUC{2502}{\sphinxunichar{2502}}
  \sphinxDUC{2514}{\sphinxunichar{2514}}
  \sphinxDUC{251C}{\sphinxunichar{251C}}
  \sphinxDUC{2572}{\textbackslash}
\fi
\usepackage{cmap}
\usepackage[T1]{fontenc}
\usepackage{amsmath,amssymb,amstext}
\usepackage{babel}



\usepackage{tgtermes}
\usepackage{tgheros}
\renewcommand{\ttdefault}{txtt}



\usepackage[Sonny]{fncychap}
\ChNameVar{\Large\normalfont\sffamily}
\ChTitleVar{\Large\normalfont\sffamily}
\usepackage{sphinx}

\fvset{fontsize=auto}
\usepackage{geometry}


% Include hyperref last.
\usepackage{hyperref}
% Fix anchor placement for figures with captions.
\usepackage{hypcap}% it must be loaded after hyperref.
% Set up styles of URL: it should be placed after hyperref.
\urlstyle{same}

\addto\captionsenglish{\renewcommand{\contentsname}{Contents:}}

\usepackage{sphinxmessages}
\setcounter{tocdepth}{1}



\title{Evolution Documentation}
\date{Jun 13, 2022}
\release{0}
\author{Louis Jourdain, Mathieu Dehouck}
\newcommand{\sphinxlogo}{\vbox{}}
\renewcommand{\releasename}{Release}
\makeindex
\begin{document}

\ifdefined\shorthandoff
  \ifnum\catcode`\=\string=\active\shorthandoff{=}\fi
  \ifnum\catcode`\"=\active\shorthandoff{"}\fi
\fi

\pagestyle{empty}
\sphinxmaketitle
\pagestyle{plain}
\sphinxtableofcontents
\pagestyle{normal}
\phantomsection\label{\detokenize{index::doc}}



\chapter{Indices and tables}
\label{\detokenize{index:indices-and-tables}}\begin{itemize}
\item {} 
\sphinxAtStartPar
\DUrole{xref,std,std-ref}{genindex}

\item {} 
\sphinxAtStartPar
\DUrole{xref,std,std-ref}{modindex}

\item {} 
\sphinxAtStartPar
\DUrole{xref,std,std-ref}{search}

\end{itemize}
\phantomsection\label{\detokenize{index:module-IPA}}\index{module@\spxentry{module}!IPA@\spxentry{IPA}}\index{IPA@\spxentry{IPA}!module@\spxentry{module}}
\sphinxAtStartPar
Created on Thu May  5 00:01:15 2022

\sphinxAtStartPar
@author: 3b13j
\index{IPA (class in IPA)@\spxentry{IPA}\spxextra{class in IPA}}

\begin{fulllineitems}
\phantomsection\label{\detokenize{index:IPA.IPA}}\pysigline{\sphinxbfcode{\sphinxupquote{class }}\sphinxcode{\sphinxupquote{IPA.}}\sphinxbfcode{\sphinxupquote{IPA}}}
\sphinxAtStartPar
A singleton class to represent the IPA viewed as a set of archetypal phonemes.

\sphinxAtStartPar
...
\begin{description}
\item[{cfeatures or vfeatures}] \leavevmode{[}list{]}
\sphinxAtStartPar
list of the str corresponding to the name of the features used in this IPA to describe the phonemes . it describe their semantics

\item[{phonemes}] \leavevmode{[}list{]}
\sphinxAtStartPar
list of archetypes objects , representing the canonical phonemes of the IPA

\item[{alphabet}] \leavevmode{[}dic{]}
\sphinxAtStartPar
dictionnary mapping an ipa character to the phoneme it describes

\item[{f2ipa}] \leavevmode{[}dic{]}
\sphinxAtStartPar
dictionnary mapping a feature list to its ipa character

\item[{classes}] \leavevmode{[}list{]}
\sphinxAtStartPar
list of the name of the different natural classes we consider in our phonology (built using a csv file)

\item[{dic\_class}] \leavevmode{[}dic{]}
\sphinxAtStartPar
dictionnary mapping a class name to the class object that represents it

\end{description}

\sphinxAtStartPar
\_\_init\_\_() no argument, automatically generates the IPA from a CSV file we created
\index{get\_IPA() (IPA.IPA static method)@\spxentry{get\_IPA()}\spxextra{IPA.IPA static method}}

\begin{fulllineitems}
\phantomsection\label{\detokenize{index:IPA.IPA.get_IPA}}\pysiglinewithargsret{\sphinxbfcode{\sphinxupquote{static }}\sphinxbfcode{\sphinxupquote{get\_IPA}}}{}{}
\sphinxAtStartPar
Static access method.

\end{fulllineitems}

\index{get\_char() (IPA.IPA method)@\spxentry{get\_char()}\spxextra{IPA.IPA method}}

\begin{fulllineitems}
\phantomsection\label{\detokenize{index:IPA.IPA.get_char}}\pysiglinewithargsret{\sphinxbfcode{\sphinxupquote{get\_char}}}{\emph{\DUrole{n}{phon}}, \emph{\DUrole{n}{verbose}\DUrole{o}{=}\DUrole{default_value}{False}}}{}
\sphinxAtStartPar
returns a string representing the input phoneme\textquotesingle{}s features
\begin{description}
\item[{phon}] \leavevmode{[}{]}
\sphinxAtStartPar
A phoneme

\item[{verbose}] \leavevmode{[}bool, optional{]}
\sphinxAtStartPar
As usual. Verbose with me means verry verbose. The default is False.

\end{description}

\sphinxAtStartPar
string

\end{fulllineitems}


\end{fulllineitems}

\index{archetype (class in IPA)@\spxentry{archetype}\spxextra{class in IPA}}

\begin{fulllineitems}
\phantomsection\label{\detokenize{index:IPA.archetype}}\pysiglinewithargsret{\sphinxbfcode{\sphinxupquote{class }}\sphinxcode{\sphinxupquote{IPA.}}\sphinxbfcode{\sphinxupquote{archetype}}}{\emph{\DUrole{n}{string}}, \emph{\DUrole{n}{feats}}, \emph{\DUrole{n}{vow}}}{}
\sphinxAtStartPar
A class to represent an archetypal phoneme (phoneme belonging to the IPA)
These objects are not mutable

\sphinxAtStartPar
...
\begin{description}
\item[{ipa}] \leavevmode{[}str{]}
\sphinxAtStartPar
character representing the phoneme in the IPA

\end{description}

\sphinxAtStartPar
features : tuple in a format we defined that represent key informations to define a phoneme\textquotesingle{}s property\textquotesingle{}
description : str
\begin{quote}

\sphinxAtStartPar
precise description of the phoneme (obtained using the IPA python module)
\end{quote}
\begin{description}
\item[{\_\_init\_\_(str, liste) }] \leavevmode
\sphinxAtStartPar
the constructor, that takes the ipa and features as attributes

\end{description}

\sphinxAtStartPar
is\_Vowel
is\_Consonant
get\_one
\index{get\_one() (IPA.archetype method)@\spxentry{get\_one()}\spxextra{IPA.archetype method}}

\begin{fulllineitems}
\phantomsection\label{\detokenize{index:IPA.archetype.get_one}}\pysiglinewithargsret{\sphinxbfcode{\sphinxupquote{get\_one}}}{\emph{\DUrole{n}{extra\_feats}}, \emph{\DUrole{n}{syllabic}}}{}
\sphinxAtStartPar
I\textquotesingle{}ll change this eventually

\end{fulllineitems}


\end{fulllineitems}

\index{cons\_dist() (in module IPA)@\spxentry{cons\_dist()}\spxextra{in module IPA}}

\begin{fulllineitems}
\phantomsection\label{\detokenize{index:IPA.cons_dist}}\pysiglinewithargsret{\sphinxcode{\sphinxupquote{IPA.}}\sphinxbfcode{\sphinxupquote{cons\_dist}}}{\emph{\DUrole{n}{features}}, \emph{\DUrole{n}{consonants}}}{}
\sphinxAtStartPar
computes the distance between a cons features and a set of consonants

\end{fulllineitems}

\index{vowel\_dist() (in module IPA)@\spxentry{vowel\_dist()}\spxextra{in module IPA}}

\begin{fulllineitems}
\phantomsection\label{\detokenize{index:IPA.vowel_dist}}\pysiglinewithargsret{\sphinxcode{\sphinxupquote{IPA.}}\sphinxbfcode{\sphinxupquote{vowel\_dist}}}{\emph{\DUrole{n}{features}}, \emph{\DUrole{n}{vowels}}}{}
\sphinxAtStartPar
computes the distance between a vow features and a set of vowells

\end{fulllineitems}

\phantomsection\label{\detokenize{index:module-Phoneme}}\index{module@\spxentry{module}!Phoneme@\spxentry{Phoneme}}\index{Phoneme@\spxentry{Phoneme}!module@\spxentry{module}}
\sphinxAtStartPar
Created on Tue May  3 00:08:43 2022

\sphinxAtStartPar
@author: 3b13j
\index{Consonant (class in Phoneme)@\spxentry{Consonant}\spxextra{class in Phoneme}}

\begin{fulllineitems}
\phantomsection\label{\detokenize{index:Phoneme.Consonant}}\pysiglinewithargsret{\sphinxbfcode{\sphinxupquote{class }}\sphinxcode{\sphinxupquote{Phoneme.}}\sphinxbfcode{\sphinxupquote{Consonant}}}{\emph{\DUrole{n}{features}}, \emph{\DUrole{n}{syllabic}}, \emph{\DUrole{n}{speller}}}{}
\sphinxAtStartPar
A class representing a Consonant

\sphinxAtStartPar
Semantic of a feature ;

\sphinxAtStartPar
syl : field of the Phoneme
voice : filed of the Phoneme
\begin{description}
\item[{Features :}] \leavevmode\begin{description}
\item[{First tuple :}] \leavevmode
\sphinxAtStartPar
0 : 
"place of articulation" : int 
1 : list of 5 manner of articulation , each coded by a boolean
\begin{quote}

\sphinxAtStartPar
"plosive"
"fricative"
"nasal"
"trill"
"lateral"
\end{quote}

\end{description}

\sphinxAtStartPar
Second tuple :
\begin{quote}

\sphinxAtStartPar
"secondary place of articulation"  int (3 or 4 possibilities, same semantics as in place of articulation)
"pren\_nasal"  bool
"aspiration" bool
\end{quote}

\end{description}

\sphinxAtStartPar
...
\begin{description}
\item[{ipa}] \leavevmode{[}str{]}
\sphinxAtStartPar
the ipa representation of a phoneme

\item[{features}] \leavevmode{[}list{]}
\sphinxAtStartPar
a list representing the features associated to the phonem

\item[{description}] \leavevmode{[}str{]}
\sphinxAtStartPar
the verbose descritption of the phoneme

\item[{voice}] \leavevmode{[}bool {]}
\sphinxAtStartPar
states if the phoneme is voiced or not

\item[{syl}] \leavevmode{[}bool {]}
\sphinxAtStartPar
states if the phoneme is center of a syllable

\end{description}

\sphinxAtStartPar
\_\_init\_\_() constructor taking all these information as input

\sphinxAtStartPar
update\_IPA  : finds the closest ipa character to represent a new phoneme

\sphinxAtStartPar
is\_round
is\_nasal
is\_palatal

\sphinxAtStartPar
get\_place
get\_manner

\sphinxAtStartPar
linearize : get the representation of the phoneme as a list of integers

\end{fulllineitems}

\index{Phoneme (class in Phoneme)@\spxentry{Phoneme}\spxextra{class in Phoneme}}

\begin{fulllineitems}
\phantomsection\label{\detokenize{index:Phoneme.Phoneme}}\pysiglinewithargsret{\sphinxbfcode{\sphinxupquote{class }}\sphinxcode{\sphinxupquote{Phoneme.}}\sphinxbfcode{\sphinxupquote{Phoneme}}}{\emph{\DUrole{n}{features}}, \emph{\DUrole{n}{syllabic}}, \emph{\DUrole{n}{speller}\DUrole{o}{=}\DUrole{default_value}{None}}}{}
\sphinxAtStartPar
A class representing a phoneme

\sphinxAtStartPar
...
\begin{description}
\item[{ipa}] \leavevmode{[}str{]}
\sphinxAtStartPar
the ipa representation of a phoneme

\item[{features}] \leavevmode{[}list{]}
\sphinxAtStartPar
a list representing the features associated to the phonem

\item[{description}] \leavevmode{[}str{]}
\sphinxAtStartPar
the verbose descritption of the phoneme

\item[{voice}] \leavevmode{[}bool {]}
\sphinxAtStartPar
states if the phoneme is voiced or not

\item[{syl}] \leavevmode{[}bool {]}
\sphinxAtStartPar
states if the phoneme is center of a syllable

\end{description}

\sphinxAtStartPar
\_\_init\_\_() constructor taking all these information as input

\sphinxAtStartPar
update\_IPA  : finds the closest ipa character to represent a new phoneme
\index{update\_IPA() (Phoneme.Phoneme method)@\spxentry{update\_IPA()}\spxextra{Phoneme.Phoneme method}}

\begin{fulllineitems}
\phantomsection\label{\detokenize{index:Phoneme.Phoneme.update_IPA}}\pysiglinewithargsret{\sphinxbfcode{\sphinxupquote{update\_IPA}}}{\emph{\DUrole{n}{config}}, \emph{\DUrole{n}{verbose}\DUrole{o}{=}\DUrole{default_value}{False}}}{}
\sphinxAtStartPar
updates the ipa field of the Phoneme it is applied to.
\begin{description}
\item[{config}] \leavevmode{[}list{]}
\sphinxAtStartPar
A list representingthe feature we want

\item[{verbose}] \leavevmode{[}bool, optional{]}
\sphinxAtStartPar
As usual. Verbose with me means verry verbose. The default is False.

\end{description}

\sphinxAtStartPar
None.

\end{fulllineitems}


\end{fulllineitems}

\index{Vowel (class in Phoneme)@\spxentry{Vowel}\spxextra{class in Phoneme}}

\begin{fulllineitems}
\phantomsection\label{\detokenize{index:Phoneme.Vowel}}\pysiglinewithargsret{\sphinxbfcode{\sphinxupquote{class }}\sphinxcode{\sphinxupquote{Phoneme.}}\sphinxbfcode{\sphinxupquote{Vowel}}}{\emph{\DUrole{n}{features}}, \emph{\DUrole{n}{syllabic}}, \emph{\DUrole{n}{speller}}}{}
\sphinxAtStartPar
A class representing a Vowel

\sphinxAtStartPar
Semantic of a feature ;

\sphinxAtStartPar
syl : field of the Phoneme
voice : filed of the Phoneme
\begin{description}
\item[{Features :}] \leavevmode
\sphinxAtStartPar
First tuple :
\begin{quote}

\sphinxAtStartPar
"fronting" : int  btw 0 and 2
"height", int btw 0 and 6
\end{quote}

\sphinxAtStartPar
Second tuple :
\begin{quote}

\sphinxAtStartPar
"round" : bool
"nasal" : bool
\end{quote}

\end{description}

\sphinxAtStartPar
...
\begin{description}
\item[{ipa}] \leavevmode{[}str{]}
\sphinxAtStartPar
the ipa representation of a phoneme

\item[{features}] \leavevmode{[}list{]}
\sphinxAtStartPar
a list representing the features associated to the phonem

\item[{description}] \leavevmode{[}str{]}
\sphinxAtStartPar
the verbose descritption of the phoneme

\item[{voice}] \leavevmode{[}bool {]}
\sphinxAtStartPar
states if the phoneme is voiced or not

\item[{syl}] \leavevmode{[}bool {]}
\sphinxAtStartPar
states if the phoneme is center of a syllable

\end{description}

\sphinxAtStartPar
\_\_init\_\_() constructor taking all these information as input

\sphinxAtStartPar
update\_IPA  : finds the closest ipa character to represent a new phoneme

\sphinxAtStartPar
get\_height
get\_front

\sphinxAtStartPar
is\_round
is\_nasal
is\_palatal
is\_voiced

\end{fulllineitems}

\index{get\_phon() (in module Phoneme)@\spxentry{get\_phon()}\spxextra{in module Phoneme}}

\begin{fulllineitems}
\phantomsection\label{\detokenize{index:Phoneme.get_phon}}\pysiglinewithargsret{\sphinxcode{\sphinxupquote{Phoneme.}}\sphinxbfcode{\sphinxupquote{get\_phon}}}{\emph{\DUrole{n}{string}}}{}
\sphinxAtStartPar
transform a string (we excpect the user to enter an ipa character) into the Phoneme object representing this character

\end{fulllineitems}

\phantomsection\label{\detokenize{index:module-Change}}\index{module@\spxentry{module}!Change@\spxentry{Change}}\index{Change@\spxentry{Change}!module@\spxentry{module}}
\sphinxAtStartPar
Created on Thu May  5 10:22:10 2022

\sphinxAtStartPar
@author: 3b13j
\index{Change (class in Change)@\spxentry{Change}\spxextra{class in Change}}

\begin{fulllineitems}
\phantomsection\label{\detokenize{index:Change.Change}}\pysiglinewithargsret{\sphinxbfcode{\sphinxupquote{class }}\sphinxcode{\sphinxupquote{Change.}}\sphinxbfcode{\sphinxupquote{Change}}}{\emph{\DUrole{n}{target}}, \emph{\DUrole{n}{effect}}, \emph{\DUrole{n}{conditions}}}{}
\sphinxAtStartPar
An abstract class representing a phonetic change.
They can be of three kinds :
\begin{itemize}
\item {} 
\sphinxAtStartPar
Phonetics ({\color{red}\bfseries{}P\_})

\item {} 
\sphinxAtStartPar
Syllabic ({\color{red}\bfseries{}I\_})

\item {} 
\sphinxAtStartPar
Wordzddzqd (?\_)

\end{itemize}

\sphinxAtStartPar
...
\begin{description}
\item[{conditions}] \leavevmode{[}list{]}
\sphinxAtStartPar
list of conditions required for the change to be applied

\item[{impacted\_phonems}] \leavevmode{[}dic{]}
\sphinxAtStartPar
dictionnary stocking the phonemes that have been impacted by a change during its application
\#TODO check it works

\item[{target}] \leavevmode{[}condition{]}
\sphinxAtStartPar
a special kind of condition that contraint the category of the phoneme that undergo the change

\end{description}

\sphinxAtStartPar
\_\_init\_\_() constructor taking all these information as input

\sphinxAtStartPar
add\_condition
set\_target
check : checks if all the conditions are satisfied before the application of the change
\index{add\_condition() (Change.Change method)@\spxentry{add\_condition()}\spxextra{Change.Change method}}

\begin{fulllineitems}
\phantomsection\label{\detokenize{index:Change.Change.add_condition}}\pysiglinewithargsret{\sphinxbfcode{\sphinxupquote{add\_condition}}}{\emph{\DUrole{n}{condition}}}{}
\sphinxAtStartPar
used to add a condition to an already built change

\end{fulllineitems}

\index{apply\_syl() (Change.Change method)@\spxentry{apply\_syl()}\spxextra{Change.Change method}}

\begin{fulllineitems}
\phantomsection\label{\detokenize{index:Change.Change.apply_syl}}\pysiglinewithargsret{\sphinxbfcode{\sphinxupquote{apply\_syl}}}{\emph{\DUrole{n}{syl}}}{}
\sphinxAtStartPar
Apply the change to a syllable

\sphinxAtStartPar
syl : Syllable

\sphinxAtStartPar
nsyl : Syllable

\end{fulllineitems}

\index{apply\_word() (Change.Change method)@\spxentry{apply\_word()}\spxextra{Change.Change method}}

\begin{fulllineitems}
\phantomsection\label{\detokenize{index:Change.Change.apply_word}}\pysiglinewithargsret{\sphinxbfcode{\sphinxupquote{apply\_word}}}{\emph{\DUrole{n}{word}}}{}
\sphinxAtStartPar
Apply the change to a word

\sphinxAtStartPar
word : Word

\sphinxAtStartPar
nword : a new word with the change applied

\end{fulllineitems}

\index{check() (Change.Change method)@\spxentry{check()}\spxextra{Change.Change method}}

\begin{fulllineitems}
\phantomsection\label{\detokenize{index:Change.Change.check}}\pysiglinewithargsret{\sphinxbfcode{\sphinxupquote{check}}}{\emph{\DUrole{n}{phon}}, \emph{\DUrole{n}{index}}, \emph{\DUrole{n}{word}}, \emph{\DUrole{n}{verbose}\DUrole{o}{=}\DUrole{default_value}{False}}}{}
\sphinxAtStartPar
check if a Change can be applied or not
\begin{description}
\item[{phon}] \leavevmode{[}Phoneme{]}
\sphinxAtStartPar
phoneme we want to apply a change on

\end{description}

\sphinxAtStartPar
index :int.
word : Word
\begin{quote}

\sphinxAtStartPar
Context
\end{quote}
\begin{description}
\item[{verbose}] \leavevmode{[}TYPE, optional{]}
\sphinxAtStartPar
DESCRIPTION. The default is False.

\end{description}
\begin{description}
\item[{apply}] \leavevmode{[}bool {]}
\sphinxAtStartPar
whether the change can be applied or not

\end{description}

\end{fulllineitems}


\end{fulllineitems}

\index{D\_change (class in Change)@\spxentry{D\_change}\spxextra{class in Change}}

\begin{fulllineitems}
\phantomsection\label{\detokenize{index:Change.D_change}}\pysiglinewithargsret{\sphinxbfcode{\sphinxupquote{class }}\sphinxcode{\sphinxupquote{Change.}}\sphinxbfcode{\sphinxupquote{D\_change}}}{\emph{\DUrole{n}{target}}, \emph{\DUrole{n}{conditions}}}{}
\sphinxAtStartPar
class allowing deletion of phonemes (monophtongisation a.o.)
\index{apply\_word() (Change.D\_change method)@\spxentry{apply\_word()}\spxextra{Change.D\_change method}}

\begin{fulllineitems}
\phantomsection\label{\detokenize{index:Change.D_change.apply_word}}\pysiglinewithargsret{\sphinxbfcode{\sphinxupquote{apply\_word}}}{\emph{\DUrole{n}{wd}}}{}
\sphinxAtStartPar
Apply the change to a word

\sphinxAtStartPar
word : Word

\sphinxAtStartPar
nword : a new word with the change applied

\end{fulllineitems}


\end{fulllineitems}

\index{I\_change (class in Change)@\spxentry{I\_change}\spxextra{class in Change}}

\begin{fulllineitems}
\phantomsection\label{\detokenize{index:Change.I_change}}\pysiglinewithargsret{\sphinxbfcode{\sphinxupquote{class }}\sphinxcode{\sphinxupquote{Change.}}\sphinxbfcode{\sphinxupquote{I\_change}}}{\emph{\DUrole{n}{target}}, \emph{\DUrole{n}{effects}}, \emph{\DUrole{n}{conditions}}}{}
\sphinxAtStartPar
class for insertion changes (diphtongisation)
\index{apply\_phon() (Change.I\_change method)@\spxentry{apply\_phon()}\spxextra{Change.I\_change method}}

\begin{fulllineitems}
\phantomsection\label{\detokenize{index:Change.I_change.apply_phon}}\pysiglinewithargsret{\sphinxbfcode{\sphinxupquote{apply\_phon}}}{\emph{\DUrole{n}{phon}}, \emph{\DUrole{n}{syl}}, \emph{\DUrole{n}{word}}}{}
\sphinxAtStartPar
we always return a list, a singleton list if the change does not apply, a pair of more otherwise

\end{fulllineitems}

\index{apply\_syl() (Change.I\_change method)@\spxentry{apply\_syl()}\spxextra{Change.I\_change method}}

\begin{fulllineitems}
\phantomsection\label{\detokenize{index:Change.I_change.apply_syl}}\pysiglinewithargsret{\sphinxbfcode{\sphinxupquote{apply\_syl}}}{\emph{\DUrole{n}{syl}}, \emph{\DUrole{n}{word}}}{}
\sphinxAtStartPar
Apply the change to a syllable

\sphinxAtStartPar
syl : Syllable

\sphinxAtStartPar
nsyl : Syllable

\end{fulllineitems}

\index{apply\_word() (Change.I\_change method)@\spxentry{apply\_word()}\spxextra{Change.I\_change method}}

\begin{fulllineitems}
\phantomsection\label{\detokenize{index:Change.I_change.apply_word}}\pysiglinewithargsret{\sphinxbfcode{\sphinxupquote{apply\_word}}}{\emph{\DUrole{n}{word}}}{}
\sphinxAtStartPar
Apply the change to a word

\sphinxAtStartPar
word : Word

\sphinxAtStartPar
nword : a new word with the change applied

\end{fulllineitems}

\index{check() (Change.I\_change method)@\spxentry{check()}\spxextra{Change.I\_change method}}

\begin{fulllineitems}
\phantomsection\label{\detokenize{index:Change.I_change.check}}\pysiglinewithargsret{\sphinxbfcode{\sphinxupquote{check}}}{\emph{\DUrole{n}{phon}}, \emph{\DUrole{n}{index}}, \emph{\DUrole{n}{word}}, \emph{\DUrole{n}{verbose}\DUrole{o}{=}\DUrole{default_value}{False}}}{}
\sphinxAtStartPar
check if a Change can be applied or not
\begin{description}
\item[{phon}] \leavevmode{[}Phoneme{]}
\sphinxAtStartPar
phoneme we want to apply a change on

\end{description}

\sphinxAtStartPar
index :int.
word : Word
\begin{quote}

\sphinxAtStartPar
Context
\end{quote}
\begin{description}
\item[{verbose}] \leavevmode{[}TYPE, optional{]}
\sphinxAtStartPar
DESCRIPTION. The default is False.

\end{description}
\begin{description}
\item[{apply}] \leavevmode{[}bool {]}
\sphinxAtStartPar
whether the change can be applied or not

\end{description}

\end{fulllineitems}


\end{fulllineitems}

\index{P\_change (class in Change)@\spxentry{P\_change}\spxextra{class in Change}}

\begin{fulllineitems}
\phantomsection\label{\detokenize{index:Change.P_change}}\pysiglinewithargsret{\sphinxbfcode{\sphinxupquote{class }}\sphinxcode{\sphinxupquote{Change.}}\sphinxbfcode{\sphinxupquote{P\_change}}}{\emph{\DUrole{n}{target}}, \emph{\DUrole{n}{effect}}, \emph{\DUrole{n}{conditions}\DUrole{o}{=}\DUrole{default_value}{None}}}{}
\sphinxAtStartPar
subclass of change 
A class modelling a phonological change .

\sphinxAtStartPar
...
\begin{description}
\item[{conditiosn}] \leavevmode{[}list{]}
\sphinxAtStartPar
list of the conditions that need to be satisfied for the change to be applied

\item[{config\_initiale}] \leavevmode{[}configuration{]}
\sphinxAtStartPar
template of the configuration selecting the feature(s) to be modified

\item[{config\_finale}] \leavevmode{[}configuration{]}
\sphinxAtStartPar
template with the modification applied

\end{description}

\sphinxAtStartPar
\_\_init\_\_() constructor taking all these information as input
\begin{description}
\item[{apply\_phon :}] \leavevmode
\sphinxAtStartPar
input : a phoneme (among other onformations)
outputs a phoneme with the change umpload

\end{description}

\sphinxAtStartPar
variants :

\sphinxAtStartPar
apply\_word
apply\_syl
apply\_language
\begin{description}
\item[{add\_condition :}] \leavevmode
\sphinxAtStartPar
input : a condition
adds it to the condition list.

\end{description}

\sphinxAtStartPar
rd\_change ; creates a random change
\index{applicable() (Change.P\_change method)@\spxentry{applicable()}\spxextra{Change.P\_change method}}

\begin{fulllineitems}
\phantomsection\label{\detokenize{index:Change.P_change.applicable}}\pysiglinewithargsret{\sphinxbfcode{\sphinxupquote{applicable}}}{\emph{\DUrole{n}{language}}}{}
\sphinxAtStartPar
checks if a change would modufy a language given as input. 
if that s not the case, it is not usefull to apply it.
\begin{description}
\item[{language}] \leavevmode{[}Language{]}
\sphinxAtStartPar
the language we would like to apply the change on

\end{description}
\begin{description}
\item[{bool}] \leavevmode
\sphinxAtStartPar
DESCRIPTION.

\end{description}

\end{fulllineitems}

\index{apply\_language() (Change.P\_change method)@\spxentry{apply\_language()}\spxextra{Change.P\_change method}}

\begin{fulllineitems}
\phantomsection\label{\detokenize{index:Change.P_change.apply_language}}\pysiglinewithargsret{\sphinxbfcode{\sphinxupquote{apply\_language}}}{\emph{\DUrole{n}{lang}}, \emph{\DUrole{n}{verbose}\DUrole{o}{=}\DUrole{default_value}{False}}}{}
\sphinxAtStartPar
Apply the change on every word in the langugage

\sphinxAtStartPar
lang : language

\sphinxAtStartPar
a new language with the change applied on every of its word

\end{fulllineitems}

\index{apply\_phon() (Change.P\_change method)@\spxentry{apply\_phon()}\spxextra{Change.P\_change method}}

\begin{fulllineitems}
\phantomsection\label{\detokenize{index:Change.P_change.apply_phon}}\pysiglinewithargsret{\sphinxbfcode{\sphinxupquote{apply\_phon}}}{\emph{\DUrole{n}{phon}}, \emph{\DUrole{n}{index}}, \emph{\DUrole{n}{word}}, \emph{\DUrole{n}{verbose}\DUrole{o}{=}\DUrole{default_value}{False}}}{}
\sphinxAtStartPar
Applies the change on a phoneme
\begin{description}
\item[{phon}] \leavevmode{[}phoneme{]}
\sphinxAtStartPar
phone

\item[{index}] \leavevmode{[}int {]}
\sphinxAtStartPar
rank of the phoneme in the overall word

\item[{word}] \leavevmode{[}word {]}
\sphinxAtStartPar
wird that encompass the phoneme we are studying, plays the role of a contest

\item[{verbose}] \leavevmode{[}bool, optional{]}
\sphinxAtStartPar
Enable or disable the verbose mode . The default is True.

\end{description}
\begin{description}
\item[{index}] \leavevmode
\sphinxAtStartPar
the updated index at the end of the process

\item[{phon }] \leavevmode
\sphinxAtStartPar
the new phoneme obtained after the applciation of the change

\end{description}

\end{fulllineitems}

\index{apply\_syl() (Change.P\_change method)@\spxentry{apply\_syl()}\spxextra{Change.P\_change method}}

\begin{fulllineitems}
\phantomsection\label{\detokenize{index:Change.P_change.apply_syl}}\pysiglinewithargsret{\sphinxbfcode{\sphinxupquote{apply\_syl}}}{\emph{\DUrole{n}{syl}}, \emph{\DUrole{n}{index}}, \emph{\DUrole{n}{wd}}, \emph{\DUrole{n}{verbose}\DUrole{o}{=}\DUrole{default_value}{False}}}{}
\sphinxAtStartPar
Apply the change to a Syllable
\begin{description}
\item[{syl}] \leavevmode{[}syllable{]}
\sphinxAtStartPar
the syllable we are going to apply the change to

\item[{index}] \leavevmode{[}int{]}
\sphinxAtStartPar
index encoding the phoneme that will undergo the change

\item[{wd}] \leavevmode{[}TYPE{]}
\sphinxAtStartPar
word of origin of the syllable, plays the role of the context

\end{description}
\begin{description}
\item[{syl}] \leavevmode{[}{]}
\sphinxAtStartPar
the updated syllable

\item[{index}] \leavevmode{[}int{]}
\sphinxAtStartPar
the index updated

\end{description}

\end{fulllineitems}

\index{apply\_word() (Change.P\_change method)@\spxentry{apply\_word()}\spxextra{Change.P\_change method}}

\begin{fulllineitems}
\phantomsection\label{\detokenize{index:Change.P_change.apply_word}}\pysiglinewithargsret{\sphinxbfcode{\sphinxupquote{apply\_word}}}{\emph{\DUrole{n}{wd}}, \emph{\DUrole{n}{verbose}\DUrole{o}{=}\DUrole{default_value}{False}}}{}
\sphinxAtStartPar
Apply the change to a word

\sphinxAtStartPar
wd : word

\sphinxAtStartPar
word : a new word with the change applied

\end{fulllineitems}

\index{just\_transform() (Change.P\_change method)@\spxentry{just\_transform()}\spxextra{Change.P\_change method}}

\begin{fulllineitems}
\phantomsection\label{\detokenize{index:Change.P_change.just_transform}}\pysiglinewithargsret{\sphinxbfcode{\sphinxupquote{just\_transform}}}{\emph{\DUrole{n}{phon}}}{}
\sphinxAtStartPar
Applies a change to transform a phoneme whithout taking care of any kind of condition

\sphinxAtStartPar
phon : Phoneme

\sphinxAtStartPar
new\_phon : Phoneme

\end{fulllineitems}


\end{fulllineitems}

\index{S\_change (class in Change)@\spxentry{S\_change}\spxextra{class in Change}}

\begin{fulllineitems}
\phantomsection\label{\detokenize{index:Change.S_change}}\pysiglinewithargsret{\sphinxbfcode{\sphinxupquote{class }}\sphinxcode{\sphinxupquote{Change.}}\sphinxbfcode{\sphinxupquote{S\_change}}}{\emph{\DUrole{n}{config\_initiale}}, \emph{\DUrole{n}{config\_finale}}, \emph{\DUrole{n}{conditions}\DUrole{o}{=}\DUrole{default_value}{{[}{]}}}}{}
\sphinxAtStartPar
subclass of change 
A class modelling a structural change in the syllable .

\sphinxAtStartPar
...
\begin{description}
\item[{conditiosn}] \leavevmode{[}list{]}
\sphinxAtStartPar
list of the conditions that need to be satisfied for the change to be applied

\item[{config\_initiale}] \leavevmode{[}list{]}
\sphinxAtStartPar
a list of 3 boleans coding stress , length and tone (None for now)

\item[{config\_finale}] \leavevmode{[}list{]}
\sphinxAtStartPar
a list of 3 boleans coding stress ,length (None for now)

\end{description}

\sphinxAtStartPar
\_\_init\_\_() constructor taking all these information as input

\sphinxAtStartPar
apply\_word
apply\_syl
apply\_lang
\begin{description}
\item[{add\_condition :}] \leavevmode
\sphinxAtStartPar
input : a condition
adds it to the condition list.

\end{description}
\index{apply\_syl() (Change.S\_change method)@\spxentry{apply\_syl()}\spxextra{Change.S\_change method}}

\begin{fulllineitems}
\phantomsection\label{\detokenize{index:Change.S_change.apply_syl}}\pysiglinewithargsret{\sphinxbfcode{\sphinxupquote{apply\_syl}}}{\emph{\DUrole{n}{syl}}, \emph{\DUrole{n}{word}}, \emph{\DUrole{n}{index}}}{}
\sphinxAtStartPar
Apply the change to a syllable

\sphinxAtStartPar
syl : Syllable

\sphinxAtStartPar
nsyl : Syllable

\end{fulllineitems}

\index{apply\_word() (Change.S\_change method)@\spxentry{apply\_word()}\spxextra{Change.S\_change method}}

\begin{fulllineitems}
\phantomsection\label{\detokenize{index:Change.S_change.apply_word}}\pysiglinewithargsret{\sphinxbfcode{\sphinxupquote{apply\_word}}}{\emph{\DUrole{n}{wd}}}{}
\sphinxAtStartPar
Apply the change to a word

\sphinxAtStartPar
word : Word

\sphinxAtStartPar
nword : a new word with the change applied

\end{fulllineitems}

\index{check() (Change.S\_change method)@\spxentry{check()}\spxextra{Change.S\_change method}}

\begin{fulllineitems}
\phantomsection\label{\detokenize{index:Change.S_change.check}}\pysiglinewithargsret{\sphinxbfcode{\sphinxupquote{check}}}{\emph{\DUrole{n}{word}}, \emph{\DUrole{n}{rank}}}{}
\sphinxAtStartPar
checks if a change can ba applied to a word
\begin{description}
\item[{word}] \leavevmode{[}word {]}
\sphinxAtStartPar
word that we test

\item[{rank}] \leavevmode{[}int{]}
\sphinxAtStartPar
rank of the phoneme

\end{description}
\begin{description}
\item[{bool}] \leavevmode
\sphinxAtStartPar
whether it is applicable

\end{description}

\end{fulllineitems}


\end{fulllineitems}

\phantomsection\label{\detokenize{index:module-Condition}}\index{module@\spxentry{module}!Condition@\spxentry{Condition}}\index{Condition@\spxentry{Condition}!module@\spxentry{module}}
\sphinxAtStartPar
Created on Thu May  5 15:09:17 2022

\sphinxAtStartPar
@author: 3b13j

\sphinxAtStartPar
Contains the condition class and some methodes that could be applied to them.
A condition is associated to a change object and states whether a change can be applied or not.
\index{Cond\_AND (class in Condition)@\spxentry{Cond\_AND}\spxextra{class in Condition}}

\begin{fulllineitems}
\phantomsection\label{\detokenize{index:Condition.Cond_AND}}\pysiglinewithargsret{\sphinxbfcode{\sphinxupquote{class }}\sphinxcode{\sphinxupquote{Condition.}}\sphinxbfcode{\sphinxupquote{Cond\_AND}}}{\emph{\DUrole{n}{conditions}}}{}
\sphinxAtStartPar
AND for conditions logic.

\end{fulllineitems}

\index{Cond\_NOT (class in Condition)@\spxentry{Cond\_NOT}\spxextra{class in Condition}}

\begin{fulllineitems}
\phantomsection\label{\detokenize{index:Condition.Cond_NOT}}\pysiglinewithargsret{\sphinxbfcode{\sphinxupquote{class }}\sphinxcode{\sphinxupquote{Condition.}}\sphinxbfcode{\sphinxupquote{Cond\_NOT}}}{\emph{\DUrole{n}{condition}}}{}
\sphinxAtStartPar
NOT for conditions logic.

\end{fulllineitems}

\index{Cond\_OR (class in Condition)@\spxentry{Cond\_OR}\spxextra{class in Condition}}

\begin{fulllineitems}
\phantomsection\label{\detokenize{index:Condition.Cond_OR}}\pysiglinewithargsret{\sphinxbfcode{\sphinxupquote{class }}\sphinxcode{\sphinxupquote{Condition.}}\sphinxbfcode{\sphinxupquote{Cond\_OR}}}{\emph{\DUrole{n}{conditions}}}{}
\sphinxAtStartPar
OR for conditions logic.

\end{fulllineitems}

\index{Condition (class in Condition)@\spxentry{Condition}\spxextra{class in Condition}}

\begin{fulllineitems}
\phantomsection\label{\detokenize{index:Condition.Condition}}\pysigline{\sphinxbfcode{\sphinxupquote{class }}\sphinxcode{\sphinxupquote{Condition.}}\sphinxbfcode{\sphinxupquote{Condition}}}
\sphinxAtStartPar
MOTHER CLASS

\sphinxAtStartPar
A class to represent a condition.
This is an abstract class.

\sphinxAtStartPar
In our implementation of phonetic change, we distinguished 3 different subclasses of conditions :
\begin{itemize}
\item {} 
\sphinxAtStartPar
phonemic condition

\item {} 
\sphinxAtStartPar
stress and syllable weight condition

\item {} 
\sphinxAtStartPar
metathesis and mechanical conditions

\end{itemize}

\end{fulllineitems}

\index{P\_condition (class in Condition)@\spxentry{P\_condition}\spxextra{class in Condition}}

\begin{fulllineitems}
\phantomsection\label{\detokenize{index:Condition.P_condition}}\pysiglinewithargsret{\sphinxbfcode{\sphinxupquote{class }}\sphinxcode{\sphinxupquote{Condition.}}\sphinxbfcode{\sphinxupquote{P\_condition}}}{\emph{\DUrole{n}{feature\_template}}, \emph{\DUrole{n}{rel\_pos}\DUrole{o}{=}\DUrole{default_value}{0}}, \emph{\DUrole{n}{absol\_pos}\DUrole{o}{=}\DUrole{default_value}{42}}, \emph{\DUrole{n}{continu}\DUrole{o}{=}\DUrole{default_value}{False}}}{}
\sphinxAtStartPar
A class to represent a condition regarding the nature of the phoneme that undergoes a change 
and its neighbours.

\sphinxAtStartPar
...
\begin{description}
\item[{template}] \leavevmode{[}list{]}
\sphinxAtStartPar
the feature template that need to be satisfied (\sphinxhyphen{}1 means  wildcard) for the condition to be satisfied

\item[{name}] \leavevmode{[}str (optionnal){]}
\sphinxAtStartPar
used to name usual changes

\item[{absol\_pos}] \leavevmode{[}int{]}
\sphinxAtStartPar
absolute offset, checks the position of the phoneme inside the word.
\sphinxhyphen{}1 means wildcard

\item[{rel\_pos}] \leavevmode{[}int{]}
\sphinxAtStartPar
defines the position of the phoneme conditionning the change regarding the phoneme undergoing it
0 means the condition applies to the phonemes that changes itself

\item[{continu}] \leavevmode{[}bool{]}\begin{description}
\item[{states if the condition needs to be satisfied by at least one of the phoneme in the range of rel\_pos or }] \leavevmode
\sphinxAtStartPar
if just the phoneme at "rel pos " is concerned.

\end{description}

\end{description}

\sphinxAtStartPar
\_\_init\_\_() constructor taking all these information as input
\begin{description}
\item[{test :}] \leavevmode
\sphinxAtStartPar
input : a word and an index.
checks whether the condition is satisfied.

\end{description}
\index{set\_absol\_pos() (Condition.P\_condition method)@\spxentry{set\_absol\_pos()}\spxextra{Condition.P\_condition method}}

\begin{fulllineitems}
\phantomsection\label{\detokenize{index:Condition.P_condition.set_absol_pos}}\pysiglinewithargsret{\sphinxbfcode{\sphinxupquote{set\_absol\_pos}}}{\emph{\DUrole{n}{value}}}{}
\sphinxAtStartPar
set the condition\textquotesingle{}s absol pos with a new value\textquotesingle{}

\sphinxAtStartPar
value : int

\sphinxAtStartPar
None.

\end{fulllineitems}

\index{set\_rel\_pos() (Condition.P\_condition method)@\spxentry{set\_rel\_pos()}\spxextra{Condition.P\_condition method}}

\begin{fulllineitems}
\phantomsection\label{\detokenize{index:Condition.P_condition.set_rel_pos}}\pysiglinewithargsret{\sphinxbfcode{\sphinxupquote{set\_rel\_pos}}}{\emph{\DUrole{n}{value}}}{}
\sphinxAtStartPar
set the condition\textquotesingle{}s relative pos with a new value\textquotesingle{}

\sphinxAtStartPar
value : int

\sphinxAtStartPar
None.

\end{fulllineitems}

\index{test() (Condition.P\_condition method)@\spxentry{test()}\spxextra{Condition.P\_condition method}}

\begin{fulllineitems}
\phantomsection\label{\detokenize{index:Condition.P_condition.test}}\pysiglinewithargsret{\sphinxbfcode{\sphinxupquote{test}}}{\emph{\DUrole{n}{word}}, \emph{\DUrole{n}{rank}}, \emph{\DUrole{n}{verbose}\DUrole{o}{=}\DUrole{default_value}{False}}}{}
\sphinxAtStartPar
Checks if a condition is satisfied on a given word.
Key method of the Condition class
\begin{description}
\item[{word}] \leavevmode{[}word{]}
\sphinxAtStartPar
word on which we check the condition

\item[{rank}] \leavevmode{[}int{]}
\sphinxAtStartPar
rank of the word we examine the condition on

\item[{verbose}] \leavevmode{[}TYPE, optional{]}
\sphinxAtStartPar
as usual. The default is False.

\end{description}

\sphinxAtStartPar
a boolean

\end{fulllineitems}


\end{fulllineitems}

\index{S\_condition (class in Condition)@\spxentry{S\_condition}\spxextra{class in Condition}}

\begin{fulllineitems}
\phantomsection\label{\detokenize{index:Condition.S_condition}}\pysiglinewithargsret{\sphinxbfcode{\sphinxupquote{class }}\sphinxcode{\sphinxupquote{Condition.}}\sphinxbfcode{\sphinxupquote{S\_condition}}}{\emph{\DUrole{n}{abs\_position}\DUrole{o}{=}\DUrole{default_value}{42}}, \emph{\DUrole{n}{rel\_pos}\DUrole{o}{=}\DUrole{default_value}{0}}, \emph{\DUrole{n}{length}\DUrole{o}{=}\DUrole{default_value}{None}}, \emph{\DUrole{n}{stress}\DUrole{o}{=}\DUrole{default_value}{None}}, \emph{\DUrole{n}{tone}\DUrole{o}{=}\DUrole{default_value}{None}}}{}
\sphinxAtStartPar
A class to represent a condition regarding the Syllabic structure of the word

\sphinxAtStartPar
...
\begin{description}
\item[{absol\_pos}] \leavevmode{[}int{]}
\sphinxAtStartPar
absolute offset, checks the position of the syllable  inside the word.
\sphinxhyphen{}1 means wildcard

\item[{rel\_pos}] \leavevmode{[}int{]}
\sphinxAtStartPar
defines the position of the phoneme conditionning the change regarding the syllable undergoing it
0 means the condition applies to the phonemes that changes itself

\item[{stress}] \leavevmode{[}bool {]}
\sphinxAtStartPar
Checks a condition on the stress

\item[{length}] \leavevmode{[}bool {]}
\sphinxAtStartPar
Checks a condition on the length

\item[{tone}] \leavevmode{[}bool {]}
\sphinxAtStartPar
Checks a condition on the tone

\end{description}

\sphinxAtStartPar
\_\_init\_\_() constructor taking all these information as input
\begin{description}
\item[{test :}] \leavevmode
\sphinxAtStartPar
input : a word and an index.
checks whether the condition is satisfied.

\end{description}
\index{test() (Condition.S\_condition method)@\spxentry{test()}\spxextra{Condition.S\_condition method}}

\begin{fulllineitems}
\phantomsection\label{\detokenize{index:Condition.S_condition.test}}\pysiglinewithargsret{\sphinxbfcode{\sphinxupquote{test}}}{\emph{\DUrole{n}{word}}, \emph{\DUrole{n}{rank}}, \emph{\DUrole{n}{verbose}\DUrole{o}{=}\DUrole{default_value}{False}}}{}
\sphinxAtStartPar
Test if the change can be applied on the word regarding the syllabic configuration
\begin{description}
\item[{word}] \leavevmode{[}word{]}
\sphinxAtStartPar
the word that undergoes the change

\item[{rank}] \leavevmode{[}int{]}
\sphinxAtStartPar
rank of the syllable that is being examined

\item[{verbose}] \leavevmode{[}bool, optional{]}
\sphinxAtStartPar
enable the verbose mode. The default is False.

\end{description}
\begin{description}
\item[{bool}] \leavevmode
\sphinxAtStartPar
boolean, if the change has to be applied

\end{description}

\end{fulllineitems}


\end{fulllineitems}

\index{rd\_p\_condition() (in module Condition)@\spxentry{rd\_p\_condition()}\spxextra{in module Condition}}

\begin{fulllineitems}
\phantomsection\label{\detokenize{index:Condition.rd_p_condition}}\pysiglinewithargsret{\sphinxcode{\sphinxupquote{Condition.}}\sphinxbfcode{\sphinxupquote{rd\_p\_condition}}}{\emph{\DUrole{n}{language}}, \emph{\DUrole{n}{rel\_pos}\DUrole{o}{=}\DUrole{default_value}{0}}, \emph{\DUrole{n}{abs\_pos}\DUrole{o}{=}\DUrole{default_value}{\sphinxhyphen{} 1}}, \emph{\DUrole{n}{continu}\DUrole{o}{=}\DUrole{default_value}{False}}}{}
\sphinxAtStartPar
generates a random P\_condition
\begin{description}
\item[{rel\_pos}] \leavevmode{[}int, optional{]}
\sphinxAtStartPar
indicates the relative position of the condition

\item[{abs\_pos}] \leavevmode{[}TYPE, int{]}
\sphinxAtStartPar
cf condition class

\item[{continu}] \leavevmode{[}TYPE, optional{]}
\sphinxAtStartPar
cf condition class

\end{description}

\sphinxAtStartPar
None.

\end{fulllineitems}

\phantomsection\label{\detokenize{index:module-Effect}}\index{module@\spxentry{module}!Effect@\spxentry{Effect}}\index{Effect@\spxentry{Effect}!module@\spxentry{module}}
\sphinxAtStartPar
Created on Wed Jun  1 09:55:33 2022

\sphinxAtStartPar
@author: 3b13j
\index{Effect (class in Effect)@\spxentry{Effect}\spxextra{class in Effect}}

\begin{fulllineitems}
\phantomsection\label{\detokenize{index:Effect.Effect}}\pysiglinewithargsret{\sphinxbfcode{\sphinxupquote{class }}\sphinxcode{\sphinxupquote{Effect.}}\sphinxbfcode{\sphinxupquote{Effect}}}{\emph{\DUrole{n}{domain}}, \emph{\DUrole{n}{effect}}}{}
\sphinxAtStartPar
A class representing the effect of a change

\sphinxAtStartPar
The effect is encoded as a dictionnary, taking as key the index of the feature to be modified, and
as value a couple {[}initial value, new value{]}
This representation will allow to modelize cyclic change.

\sphinxAtStartPar
An effect could be built using a target

\sphinxAtStartPar
...
\begin{description}
\item[{target}] \leavevmode{[}(optionnal ) tuple{]}
\sphinxAtStartPar
a feature pattern representing the phonemes the change can be applied to

\end{description}

\sphinxAtStartPar
\_\_init\_\_() constructor taking all these information as input

\sphinxAtStartPar
random\_effect

\end{fulllineitems}

\phantomsection\label{\detokenize{index:module-Generator}}\index{module@\spxentry{module}!Generator@\spxentry{Generator}}\index{Generator@\spxentry{Generator}!module@\spxentry{module}}
\sphinxAtStartPar
to fill

\phantomsection\label{\detokenize{index:module-conllu_utilities}}\index{module@\spxentry{module}!conllu\_utilities@\spxentry{conllu\_utilities}}\index{conllu\_utilities@\spxentry{conllu\_utilities}!module@\spxentry{module}}
\sphinxAtStartPar
Created on Mon May  2 16:34:02 2022

\sphinxAtStartPar
@author: 3b13j

\sphinxAtStartPar
contains usefull functions to extract data from conllu files
\index{GetKey() (in module conllu\_utilities)@\spxentry{GetKey()}\spxextra{in module conllu\_utilities}}

\begin{fulllineitems}
\phantomsection\label{\detokenize{index:conllu_utilities.GetKey}}\pysiglinewithargsret{\sphinxcode{\sphinxupquote{conllu\_utilities.}}\sphinxbfcode{\sphinxupquote{GetKey}}}{\emph{\DUrole{n}{val}}, \emph{\DUrole{n}{dic}}}{}
\sphinxAtStartPar
to get the key giving as input the value and a dictionnary

\end{fulllineitems}

\index{extract\_conllu() (in module conllu\_utilities)@\spxentry{extract\_conllu()}\spxextra{in module conllu\_utilities}}

\begin{fulllineitems}
\phantomsection\label{\detokenize{index:conllu_utilities.extract_conllu}}\pysiglinewithargsret{\sphinxcode{\sphinxupquote{conllu\_utilities.}}\sphinxbfcode{\sphinxupquote{extract\_conllu}}}{\emph{\DUrole{n}{path}}}{}
\sphinxAtStartPar
Extracts data from a conllu file
\begin{description}
\item[{path}] \leavevmode{[}str{]}
\sphinxAtStartPar
a path to the origin file

\end{description}
\begin{description}
\item[{res}] \leavevmode{[}a list {]}
\sphinxAtStartPar
the result of a successful extraction

\end{description}

\end{fulllineitems}

\index{word\_2\_phoneme\_lat() (in module conllu\_utilities)@\spxentry{word\_2\_phoneme\_lat()}\spxextra{in module conllu\_utilities}}

\begin{fulllineitems}
\phantomsection\label{\detokenize{index:conllu_utilities.word_2_phoneme_lat}}\pysiglinewithargsret{\sphinxcode{\sphinxupquote{conllu\_utilities.}}\sphinxbfcode{\sphinxupquote{word\_2\_phoneme\_lat}}}{\emph{\DUrole{n}{string}}, \emph{\DUrole{n}{alph}}}{}
\sphinxAtStartPar
A useless thing that treats latin

\end{fulllineitems}

\phantomsection\label{\detokenize{index:module-0}}\index{module@\spxentry{module}!Effect@\spxentry{Effect}}\index{Effect@\spxentry{Effect}!module@\spxentry{module}}
\sphinxAtStartPar
Created on Wed Jun  1 09:55:33 2022

\sphinxAtStartPar
@author: 3b13j
\index{Effect (class in Effect)@\spxentry{Effect}\spxextra{class in Effect}}

\begin{fulllineitems}
\phantomsection\label{\detokenize{index:id0}}\pysiglinewithargsret{\sphinxbfcode{\sphinxupquote{class }}\sphinxcode{\sphinxupquote{Effect.}}\sphinxbfcode{\sphinxupquote{Effect}}}{\emph{\DUrole{n}{domain}}, \emph{\DUrole{n}{effect}}}{}
\sphinxAtStartPar
A class representing the effect of a change

\sphinxAtStartPar
The effect is encoded as a dictionnary, taking as key the index of the feature to be modified, and
as value a couple {[}initial value, new value{]}
This representation will allow to modelize cyclic change.

\sphinxAtStartPar
An effect could be built using a target

\sphinxAtStartPar
...
\begin{description}
\item[{target}] \leavevmode{[}(optionnal ) tuple{]}
\sphinxAtStartPar
a feature pattern representing the phonemes the change can be applied to

\end{description}

\sphinxAtStartPar
\_\_init\_\_() constructor taking all these information as input

\sphinxAtStartPar
random\_effect

\end{fulllineitems}

\phantomsection\label{\detokenize{index:module-encoder_decoder}}\index{module@\spxentry{module}!encoder\_decoder@\spxentry{encoder\_decoder}}\index{encoder\_decoder@\spxentry{encoder\_decoder}!module@\spxentry{module}}
\sphinxAtStartPar
Created on Fri May 20 11:15:50 2022

\sphinxAtStartPar
@author: 3b13j
\index{decode\_f() (in module encoder\_decoder)@\spxentry{decode\_f()}\spxextra{in module encoder\_decoder}}

\begin{fulllineitems}
\phantomsection\label{\detokenize{index:encoder_decoder.decode_f}}\pysiglinewithargsret{\sphinxcode{\sphinxupquote{encoder\_decoder.}}\sphinxbfcode{\sphinxupquote{decode\_f}}}{\emph{\DUrole{n}{string}}}{}
\sphinxAtStartPar
Decode a string we encoded earlier

\sphinxAtStartPar
string : to be decoded

\end{fulllineitems}

\index{encode\_f() (in module encoder\_decoder)@\spxentry{encode\_f()}\spxextra{in module encoder\_decoder}}

\begin{fulllineitems}
\phantomsection\label{\detokenize{index:encoder_decoder.encode_f}}\pysiglinewithargsret{\sphinxcode{\sphinxupquote{encoder\_decoder.}}\sphinxbfcode{\sphinxupquote{encode\_f}}}{\emph{\DUrole{n}{feat}}}{}
\sphinxAtStartPar
encode a feature into the format we chose
\begin{description}
\item[{feat}] \leavevmode{[}list{]}
\sphinxAtStartPar
the feature rpz of a template

\end{description}

\sphinxAtStartPar
s : str

\end{fulllineitems}

\phantomsection\label{\detokenize{index:module-IPA_utils}}\index{module@\spxentry{module}!IPA\_utils@\spxentry{IPA\_utils}}\index{IPA\_utils@\spxentry{IPA\_utils}!module@\spxentry{module}}\phantomsection\label{\detokenize{index:module-Language}}\index{module@\spxentry{module}!Language@\spxentry{Language}}\index{Language@\spxentry{Language}!module@\spxentry{module}}
\sphinxAtStartPar
Created on Thu May  5 09:28:54 2022

\sphinxAtStartPar
@author: 3b13j

\sphinxAtStartPar
Contains the languge class
\index{Language (class in Language)@\spxentry{Language}\spxextra{class in Language}}

\begin{fulllineitems}
\phantomsection\label{\detokenize{index:Language.Language}}\pysiglinewithargsret{\sphinxbfcode{\sphinxupquote{class }}\sphinxcode{\sphinxupquote{Language.}}\sphinxbfcode{\sphinxupquote{Language}}}{\emph{\DUrole{n}{name}}, \emph{\DUrole{n}{dic}}}{}
\sphinxAtStartPar
A class to represent a language, considered as the list of the phonems it possesses.

\sphinxAtStartPar
...
\begin{description}
\item[{name}] \leavevmode{[}str{]}
\sphinxAtStartPar
name of the language

\item[{voc}] \leavevmode{[}dic {]}
\sphinxAtStartPar
dicitonnary storing all the words of the language

\item[{phonemes}] \leavevmode{[}list{]}
\sphinxAtStartPar
list of all the phonemes belonging to the language

\item[{dic\_phonemes}] \leavevmode{[}dic{]}
\sphinxAtStartPar
a dic mapping a character to a phoneme

\end{description}

\sphinxAtStartPar
\_\_init\_\_() the constructor
\index{compare() (Language.Language method)@\spxentry{compare()}\spxextra{Language.Language method}}

\begin{fulllineitems}
\phantomsection\label{\detokenize{index:Language.Language.compare}}\pysiglinewithargsret{\sphinxbfcode{\sphinxupquote{compare}}}{\emph{\DUrole{n}{language}}, \emph{\DUrole{n}{verbose}\DUrole{o}{=}\DUrole{default_value}{False}}}{}~\begin{description}
\item[{language}] \leavevmode{[}another Language object{]}
\sphinxAtStartPar
DESCRIPTION.

\end{description}
\begin{description}
\item[{differents}] \leavevmode{[}list {]}
\sphinxAtStartPar
the list of word that have been modified in the new language

\end{description}

\end{fulllineitems}


\end{fulllineitems}

\index{State (class in Language)@\spxentry{State}\spxextra{class in Language}}

\begin{fulllineitems}
\phantomsection\label{\detokenize{index:Language.State}}\pysiglinewithargsret{\sphinxbfcode{\sphinxupquote{class }}\sphinxcode{\sphinxupquote{Language.}}\sphinxbfcode{\sphinxupquote{State}}}{\emph{\DUrole{n}{language}}}{}
\sphinxAtStartPar
A condensed representation of a language for faster interaction and change generation

\sphinxAtStartPar
...
\begin{description}
\item[{phonemes}] \leavevmode{[}dict{]}
\sphinxAtStartPar
list of all the phonemes belonging to the language

\item[{syllables}] \leavevmode{[}dict{]}
\sphinxAtStartPar
a dict of syllables

\end{description}

\sphinxAtStartPar
\_\_init\_\_() the constructor

\end{fulllineitems}

\phantomsection\label{\detokenize{index:module-log_utilities}}\index{module@\spxentry{module}!log\_utilities@\spxentry{log\_utilities}}\index{log\_utilities@\spxentry{log\_utilities}!module@\spxentry{module}}
\sphinxAtStartPar
Created on Tue May 17 16:02:54 2022

\sphinxAtStartPar
@author: 3b13j

\sphinxAtStartPar
Contains some side methods to write the state of some objects and describe the execution of the program  in log files
\index{create\_breviary() (in module log\_utilities)@\spxentry{create\_breviary()}\spxextra{in module log\_utilities}}

\begin{fulllineitems}
\phantomsection\label{\detokenize{index:log_utilities.create_breviary}}\pysiglinewithargsret{\sphinxcode{\sphinxupquote{log\_utilities.}}\sphinxbfcode{\sphinxupquote{create\_breviary}}}{}{}
\sphinxAtStartPar
Creates a user friendly document that describes all the natural classes that exist with regard to the
IPA we use at the heart of the program""

\end{fulllineitems}

\index{extract\_changed\_words() (in module log\_utilities)@\spxentry{extract\_changed\_words()}\spxextra{in module log\_utilities}}

\begin{fulllineitems}
\phantomsection\label{\detokenize{index:log_utilities.extract_changed_words}}\pysiglinewithargsret{\sphinxcode{\sphinxupquote{log\_utilities.}}\sphinxbfcode{\sphinxupquote{extract\_changed\_words}}}{\emph{\DUrole{n}{path}}, \emph{\DUrole{n}{write}\DUrole{o}{=}\DUrole{default_value}{False}}}{}
\sphinxAtStartPar
Analyses a dictionnary log and extracts only the words that were changed.

\sphinxAtStartPar
The function the can if the user wants it write the modified words at the end of the same document
\begin{description}
\item[{path}] \leavevmode{[}str{]}
\sphinxAtStartPar
path to the file

\item[{write}] \leavevmode{[}bool{]}
\sphinxAtStartPar
Does the user want to write down the modified words at the end of the document ?

\end{description}

\sphinxAtStartPar
chg\_wds :list

\end{fulllineitems}

\index{langcomp2log() (in module log\_utilities)@\spxentry{langcomp2log()}\spxextra{in module log\_utilities}}

\begin{fulllineitems}
\phantomsection\label{\detokenize{index:log_utilities.langcomp2log}}\pysiglinewithargsret{\sphinxcode{\sphinxupquote{log\_utilities.}}\sphinxbfcode{\sphinxupquote{langcomp2log}}}{\emph{\DUrole{n}{l1}}, \emph{\DUrole{n}{l2}}, \emph{\DUrole{n}{path}}}{}
\sphinxAtStartPar
Comapres the vobulary of two languages and writes the comparison in a log (subfunction used to trace the evolution between two language state)
BE CAREFUL, we excpect the two languages to be related / at least to have the same voc size for this operation to make sense.

\sphinxAtStartPar
l1 : Language

\sphinxAtStartPar
l2 :Language
\begin{description}
\item[{path}] \leavevmode{[}str{]}
\sphinxAtStartPar
path to destination file

\end{description}

\sphinxAtStartPar
None.

\end{fulllineitems}

\index{lgs2log() (in module log\_utilities)@\spxentry{lgs2log()}\spxextra{in module log\_utilities}}

\begin{fulllineitems}
\phantomsection\label{\detokenize{index:log_utilities.lgs2log}}\pysiglinewithargsret{\sphinxcode{\sphinxupquote{log\_utilities.}}\sphinxbfcode{\sphinxupquote{lgs2log}}}{\emph{\DUrole{n}{liste}}}{}
\sphinxAtStartPar
print the evolution of a language step by step.

\sphinxAtStartPar
liste : list of languages where the i+1 th element is the result of the evolution of the ith

\sphinxAtStartPar
None.

\end{fulllineitems}

\index{phon2log() (in module log\_utilities)@\spxentry{phon2log()}\spxextra{in module log\_utilities}}

\begin{fulllineitems}
\phantomsection\label{\detokenize{index:log_utilities.phon2log}}\pysiglinewithargsret{\sphinxcode{\sphinxupquote{log\_utilities.}}\sphinxbfcode{\sphinxupquote{phon2log}}}{\emph{\DUrole{n}{phon}}, \emph{\DUrole{n}{path}}}{}
\sphinxAtStartPar
writes a phoneme in the log format we defined

\sphinxAtStartPar
phon : phoneme to write in the script
\begin{description}
\item[{path}] \leavevmode{[}str {]}
\sphinxAtStartPar
path to the target file

\end{description}

\sphinxAtStartPar
None.

\end{fulllineitems}

\index{purge\_log() (in module log\_utilities)@\spxentry{purge\_log()}\spxextra{in module log\_utilities}}

\begin{fulllineitems}
\phantomsection\label{\detokenize{index:log_utilities.purge_log}}\pysiglinewithargsret{\sphinxcode{\sphinxupquote{log\_utilities.}}\sphinxbfcode{\sphinxupquote{purge\_log}}}{\emph{\DUrole{n}{path}}}{}
\sphinxAtStartPar
Clears a log file

\sphinxAtStartPar
path :

\sphinxAtStartPar
None.

\end{fulllineitems}

\index{samples2log() (in module log\_utilities)@\spxentry{samples2log()}\spxextra{in module log\_utilities}}

\begin{fulllineitems}
\phantomsection\label{\detokenize{index:log_utilities.samples2log}}\pysiglinewithargsret{\sphinxcode{\sphinxupquote{log\_utilities.}}\sphinxbfcode{\sphinxupquote{samples2log}}}{\emph{\DUrole{n}{path}}, \emph{\DUrole{n}{liste}}, \emph{\DUrole{n}{n}\DUrole{o}{=}\DUrole{default_value}{10}}}{}
\sphinxAtStartPar
Method that writes down only some of the words modified by a change.
\begin{description}
\item[{path}] \leavevmode{[}str{]}
\sphinxAtStartPar
path to the file you want tp write in.

\end{description}

\sphinxAtStartPar
liste : list of changed words :
n : int, optional
\begin{quote}

\sphinxAtStartPar
number of words that will be printed. The default is 10.
\end{quote}

\sphinxAtStartPar
None.

\end{fulllineitems}

\index{target2str() (in module log\_utilities)@\spxentry{target2str()}\spxextra{in module log\_utilities}}

\begin{fulllineitems}
\phantomsection\label{\detokenize{index:log_utilities.target2str}}\pysiglinewithargsret{\sphinxcode{\sphinxupquote{log\_utilities.}}\sphinxbfcode{\sphinxupquote{target2str}}}{\emph{\DUrole{n}{feat}}}{}
\sphinxAtStartPar
Encode the target of a change to write it latter in the log
\begin{description}
\item[{feat}] \leavevmode{[}list {]}
\sphinxAtStartPar
description of a feature template

\end{description}

\sphinxAtStartPar
s : qtring describing it

\end{fulllineitems}

\index{write\_in\_log() (in module log\_utilities)@\spxentry{write\_in\_log()}\spxextra{in module log\_utilities}}

\begin{fulllineitems}
\phantomsection\label{\detokenize{index:log_utilities.write_in_log}}\pysiglinewithargsret{\sphinxcode{\sphinxupquote{log\_utilities.}}\sphinxbfcode{\sphinxupquote{write\_in\_log}}}{\emph{\DUrole{n}{path}}, \emph{\DUrole{n}{string}}}{}
\sphinxAtStartPar
Takes as input the name of a log file and the sentence that it sould add in it

\end{fulllineitems}

\phantomsection\label{\detokenize{index:module-Natural_class}}\index{module@\spxentry{module}!Natural\_class@\spxentry{Natural\_class}}\index{Natural\_class@\spxentry{Natural\_class}!module@\spxentry{module}}
\sphinxAtStartPar
Created on Tue May  3 12:20:06 2022

\sphinxAtStartPar
@author: 3b13j
\index{Natural\_class (class in Natural\_class)@\spxentry{Natural\_class}\spxextra{class in Natural\_class}}

\begin{fulllineitems}
\phantomsection\label{\detokenize{index:Natural_class.Natural_class}}\pysiglinewithargsret{\sphinxbfcode{\sphinxupquote{class }}\sphinxcode{\sphinxupquote{Natural\_class.}}\sphinxbfcode{\sphinxupquote{Natural\_class}}}{\emph{\DUrole{n}{name}}, \emph{\DUrole{n}{feat}}, \emph{\DUrole{n}{vow}}, \emph{\DUrole{n}{lin}}}{}
\sphinxAtStartPar
An object representing a natural class

\sphinxAtStartPar
...
\begin{description}
\item[{name}] \leavevmode{[}str {]}
\sphinxAtStartPar
The name of a class

\item[{members}] \leavevmode{[}list{]}
\sphinxAtStartPar
list of the phonemes belonging to a class

\item[{template}] \leavevmode{[}list{]}
\sphinxAtStartPar
list of the feature template representing the class. initiated with full wildcards

\end{description}

\sphinxAtStartPar
\_\_init\_\_() constructor taking all these information as input

\sphinxAtStartPar
add\_phon
set\_template

\end{fulllineitems}

\index{create\_classes() (in module Natural\_class)@\spxentry{create\_classes()}\spxextra{in module Natural\_class}}

\begin{fulllineitems}
\phantomsection\label{\detokenize{index:Natural_class.create_classes}}\pysiglinewithargsret{\sphinxcode{\sphinxupquote{Natural\_class.}}\sphinxbfcode{\sphinxupquote{create\_classes}}}{\emph{\DUrole{n}{alphabet}}}{}
\sphinxAtStartPar
creates the natural classes from an ipa alphabet
definded outside the class to be used once and for all

\sphinxAtStartPar
alphabet : an ipa alphabet
\begin{description}
\item[{dic\_class}] \leavevmode{[}dic{]}
\sphinxAtStartPar
dictionnary mapping the name of a natural class to the class object

\item[{classes}] \leavevmode{[}list{]}
\sphinxAtStartPar
list of the natural classes we want to work with

\end{description}

\end{fulllineitems}

\index{list2class() (in module Natural\_class)@\spxentry{list2class()}\spxextra{in module Natural\_class}}

\begin{fulllineitems}
\phantomsection\label{\detokenize{index:Natural_class.list2class}}\pysiglinewithargsret{\sphinxcode{\sphinxupquote{Natural\_class.}}\sphinxbfcode{\sphinxupquote{list2class}}}{\emph{\DUrole{n}{name}}, \emph{\DUrole{n}{clas}}}{}
\sphinxAtStartPar
Creates a Natural\_class with the name given as input and add all the phonemes givent in the second input (list)

\end{fulllineitems}

\phantomsection\label{\detokenize{index:module-rd_changer}}\index{module@\spxentry{module}!rd\_changer@\spxentry{rd\_changer}}\index{rd\_changer@\spxentry{rd\_changer}!module@\spxentry{module}}
\sphinxAtStartPar
Created on Tue May 17 13:04:22 2022

\sphinxAtStartPar
@author: 3b13j

\phantomsection\label{\detokenize{index:module-Sampling}}\index{module@\spxentry{module}!Sampling@\spxentry{Sampling}}\index{Sampling@\spxentry{Sampling}!module@\spxentry{module}}
\sphinxAtStartPar
Created on Tue May 31 12:42:01 2022

\sphinxAtStartPar
sampling :

\sphinxAtStartPar
@author: 3b13j
\index{MatricesC (in module Sampling)@\spxentry{MatricesC}\spxextra{in module Sampling}}

\begin{fulllineitems}
\phantomsection\label{\detokenize{index:Sampling.MatricesC}}\pysigline{\sphinxcode{\sphinxupquote{Sampling.}}\sphinxbfcode{\sphinxupquote{MatricesC}}\sphinxbfcode{\sphinxupquote{ = ((array({[}{[}0., 4., 2., 1., 0., 0., 0., 0., 0., 0., 0., 0.{]},        {[}4., 0., 4., 2., 1., 0., 0., 0., 0., 0., 0., 0.{]},        {[}2., 4., 0., 4., 2., 1., 0., 0., 0., 0., 0., 0.{]},        {[}1., 2., 4., 0., 4., 2., 1., 0., 0., 0., 0., 0.{]},        {[}0., 1., 2., 4., 0., 4., 2., 1., 0., 0., 0., 0.{]},        {[}0., 0., 1., 2., 4., 0., 4., 2., 1., 0., 0., 0.{]},        {[}0., 0., 0., 1., 2., 4., 0., 4., 2., 1., 0., 0.{]},        {[}0., 0., 0., 0., 1., 2., 4., 0., 4., 2., 1., 0.{]},        {[}0., 0., 0., 0., 0., 1., 2., 4., 0., 4., 2., 1.{]},        {[}0., 0., 0., 0., 0., 0., 1., 2., 4., 0., 4., 2.{]},        {[}0., 0., 0., 0., 0., 0., 0., 1., 2., 4., 0., 4.{]},        {[}0., 0., 0., 0., 0., 0., 0., 0., 1., 2., 4., 0.{]}{]}), array({[}{[}0, 5, 5, 5, 5, 5, 5, 1, 1, 1{]},        {[}5, 0, 5, 1, 1, 5, 1, 1, 1, 1{]},        {[}5, 5, 0, 5, 3, 3, 5, 1, 1, 5{]},        {[}5, 3, 5, 0, 5, 5, 1, 5, 1, 5{]},        {[}5, 1, 1, 3, 0, 5, 1, 1, 1, 1{]},        {[}3, 1, 1, 3, 5, 0, 1, 1, 1, 1{]},        {[}5, 1, 3, 3, 3, 3, 0, 5, 5, 1{]},        {[}1, 1, 1, 5, 1, 1, 5, 0, 1, 1{]},        {[}1, 1, 1, 1, 1, 5, 5, 1, 0, 1{]},        {[}1, 1, 5, 5, 1, 1, 1, 1, 1, 0{]}{]}), array({[}{[}0., 4.{]},        {[}4., 0.{]}{]})), (array({[}{[}0., 4., 2., 1.{]},        {[}4., 0., 4., 2.{]},        {[}2., 4., 0., 4.{]},        {[}1., 2., 4., 0.{]}{]}), array({[}{[}0., 4.{]},        {[}4., 0.{]}{]}), array({[}{[}0., 4.{]},        {[}4., 0.{]}{]})))}}}
\sphinxAtStartPar
être generique, choisir une manière

\sphinxAtStartPar
manière de remplir, vitef

\sphinxAtStartPar
mais manière de les construire
design fort,  choix qui a un sens , cohérent et on le tient , ou qqc de numérique, générique

\sphinxAtStartPar
le vecteur de feats est composé de deux bouts. 
ds ces 2 bouts, on a des index,

\sphinxAtStartPar
chaque attribut à deu

\sphinxAtStartPar
deux boucles for imbriquées. enumeration des turcs du premier niveau ,ceux du second niceau. 
faire des paires 
on tire au hasard une paire A B 
phon.feature (A B )

\sphinxAtStartPar
crer matrices,  on les met dans des structures qui ont la même forme que les features

\end{fulllineitems}

\phantomsection\label{\detokenize{index:module-Tree}}\index{module@\spxentry{module}!Tree@\spxentry{Tree}}\index{Tree@\spxentry{Tree}!module@\spxentry{module}}
\sphinxAtStartPar
Created on Wed May 18 17:00:17 2022

\sphinxAtStartPar
@author: 3b13j
\index{L\_tree (class in Tree)@\spxentry{L\_tree}\spxextra{class in Tree}}

\begin{fulllineitems}
\phantomsection\label{\detokenize{index:Tree.L_tree}}\pysiglinewithargsret{\sphinxbfcode{\sphinxupquote{class }}\sphinxcode{\sphinxupquote{Tree.}}\sphinxbfcode{\sphinxupquote{L\_tree}}}{\emph{\DUrole{n}{language}}, \emph{\DUrole{n}{parent}\DUrole{o}{=}\DUrole{default_value}{None}}}{}
\sphinxAtStartPar
a special kind of phylogentic tree storing our languages
as defined, the structure should be names "forward tree" since the change is unidirectionnal
\index{elaborate\_history\_graph() (Tree.L\_tree method)@\spxentry{elaborate\_history\_graph()}\spxextra{Tree.L\_tree method}}

\begin{fulllineitems}
\phantomsection\label{\detokenize{index:Tree.L_tree.elaborate_history_graph}}\pysiglinewithargsret{\sphinxbfcode{\sphinxupquote{elaborate\_history\_graph}}}{\emph{\DUrole{n}{word}}}{}
\sphinxAtStartPar
Extracts a subgraph representing the history of the evolution of a particular word in all the languages we generated

\end{fulllineitems}

\index{get\_ad\_2\_tree() (Tree.L\_tree method)@\spxentry{get\_ad\_2\_tree()}\spxextra{Tree.L\_tree method}}

\begin{fulllineitems}
\phantomsection\label{\detokenize{index:Tree.L_tree.get_ad_2_tree}}\pysiglinewithargsret{\sphinxbfcode{\sphinxupquote{get\_ad\_2\_tree}}}{\emph{\DUrole{n}{dic}\DUrole{o}{=}\DUrole{default_value}{\{\}}}}{}
\sphinxAtStartPar
returns a dictionnary mapping an adress to the tree object

\end{fulllineitems}

\index{get\_depth() (Tree.L\_tree method)@\spxentry{get\_depth()}\spxextra{Tree.L\_tree method}}

\begin{fulllineitems}
\phantomsection\label{\detokenize{index:Tree.L_tree.get_depth}}\pysiglinewithargsret{\sphinxbfcode{\sphinxupquote{get\_depth}}}{\emph{\DUrole{n}{dic}\DUrole{o}{=}\DUrole{default_value}{\{\}}}}{}
\sphinxAtStartPar
return a dictionarry mapping the address of a tree to its depth

\end{fulllineitems}

\index{get\_final\_state\_of\_the\_evolution() (Tree.L\_tree method)@\spxentry{get\_final\_state\_of\_the\_evolution()}\spxextra{Tree.L\_tree method}}

\begin{fulllineitems}
\phantomsection\label{\detokenize{index:Tree.L_tree.get_final_state_of_the_evolution}}\pysiglinewithargsret{\sphinxbfcode{\sphinxupquote{get\_final\_state\_of\_the\_evolution}}}{}{}
\sphinxAtStartPar
Returns the languages at the end of our evolution tree

\end{fulllineitems}

\index{get\_history\_word() (Tree.L\_tree method)@\spxentry{get\_history\_word()}\spxextra{Tree.L\_tree method}}

\begin{fulllineitems}
\phantomsection\label{\detokenize{index:Tree.L_tree.get_history_word}}\pysiglinewithargsret{\sphinxbfcode{\sphinxupquote{get\_history\_word}}}{\emph{\DUrole{n}{word}}, \emph{\DUrole{n}{liste}\DUrole{o}{=}\DUrole{default_value}{{[}{]}}}}{}
\sphinxAtStartPar
Stores the state of the word in the histtory of all the generated languages

\end{fulllineitems}

\index{get\_languages() (Tree.L\_tree method)@\spxentry{get\_languages()}\spxextra{Tree.L\_tree method}}

\begin{fulllineitems}
\phantomsection\label{\detokenize{index:Tree.L_tree.get_languages}}\pysiglinewithargsret{\sphinxbfcode{\sphinxupquote{get\_languages}}}{\emph{\DUrole{n}{liste}\DUrole{o}{=}\DUrole{default_value}{\{\}}}}{}
\sphinxAtStartPar
Returns a dictionnary mapping the adress of a tree to the language it stores

\end{fulllineitems}

\index{get\_leaves() (Tree.L\_tree method)@\spxentry{get\_leaves()}\spxextra{Tree.L\_tree method}}

\begin{fulllineitems}
\phantomsection\label{\detokenize{index:Tree.L_tree.get_leaves}}\pysiglinewithargsret{\sphinxbfcode{\sphinxupquote{get\_leaves}}}{\emph{\DUrole{n}{liste}\DUrole{o}{=}\DUrole{default_value}{None}}}{}
\sphinxAtStartPar
Get the list of the leaves of a language

\end{fulllineitems}

\index{get\_nodes() (Tree.L\_tree method)@\spxentry{get\_nodes()}\spxextra{Tree.L\_tree method}}

\begin{fulllineitems}
\phantomsection\label{\detokenize{index:Tree.L_tree.get_nodes}}\pysiglinewithargsret{\sphinxbfcode{\sphinxupquote{get\_nodes}}}{\emph{\DUrole{n}{liste}\DUrole{o}{=}\DUrole{default_value}{None}}}{}
\sphinxAtStartPar
returns a list containing all the tree object that are nodes in the mother tree

\end{fulllineitems}

\index{get\_path\_to\_root() (Tree.L\_tree method)@\spxentry{get\_path\_to\_root()}\spxextra{Tree.L\_tree method}}

\begin{fulllineitems}
\phantomsection\label{\detokenize{index:Tree.L_tree.get_path_to_root}}\pysiglinewithargsret{\sphinxbfcode{\sphinxupquote{get\_path\_to\_root}}}{}{}
\sphinxAtStartPar
Returns the list of all the nodes leading from the target node to the root of the tree

\end{fulllineitems}

\index{get\_scores() (Tree.L\_tree method)@\spxentry{get\_scores()}\spxextra{Tree.L\_tree method}}

\begin{fulllineitems}
\phantomsection\label{\detokenize{index:Tree.L_tree.get_scores}}\pysiglinewithargsret{\sphinxbfcode{\sphinxupquote{get\_scores}}}{\emph{\DUrole{n}{liste}\DUrole{o}{=}\DUrole{default_value}{{[}{]}}}, \emph{\DUrole{n}{scores}\DUrole{o}{=}\DUrole{default_value}{{[}{]}}}}{}
\sphinxAtStartPar
Get a mapping between change objects and the chance they have to appear

\end{fulllineitems}

\index{history\_to\_graph() (Tree.L\_tree method)@\spxentry{history\_to\_graph()}\spxextra{Tree.L\_tree method}}

\begin{fulllineitems}
\phantomsection\label{\detokenize{index:Tree.L_tree.history_to_graph}}\pysiglinewithargsret{\sphinxbfcode{\sphinxupquote{history\_to\_graph}}}{\emph{\DUrole{n}{word}}}{}
\sphinxAtStartPar
print a graph only displaying the informations on the evolution of a single word

\end{fulllineitems}

\index{pick\_a\_node() (Tree.L\_tree method)@\spxentry{pick\_a\_node()}\spxextra{Tree.L\_tree method}}

\begin{fulllineitems}
\phantomsection\label{\detokenize{index:Tree.L_tree.pick_a_node}}\pysiglinewithargsret{\sphinxbfcode{\sphinxupquote{pick\_a\_node}}}{}{}
\sphinxAtStartPar
Pick a random node from a tree

\end{fulllineitems}

\index{print\_history\_to\_graph() (Tree.L\_tree method)@\spxentry{print\_history\_to\_graph()}\spxextra{Tree.L\_tree method}}

\begin{fulllineitems}
\phantomsection\label{\detokenize{index:Tree.L_tree.print_history_to_graph}}\pysiglinewithargsret{\sphinxbfcode{\sphinxupquote{print\_history\_to\_graph}}}{\emph{\DUrole{n}{word}}}{}
\sphinxAtStartPar
Display a graph in which the nodes represent a word at a certain langugage state, and the edges the link between two languages

\end{fulllineitems}


\end{fulllineitems}

\phantomsection\label{\detokenize{index:module-Word}}\index{module@\spxentry{module}!Word@\spxentry{Word}}\index{Word@\spxentry{Word}!module@\spxentry{module}}
\sphinxAtStartPar
Created on Wed May  4 14:39:58 2022

\sphinxAtStartPar
@author: 3b13j

\sphinxAtStartPar
Contains the Word class and some methods used specifically to work with it
\index{Syllable (class in Word)@\spxentry{Syllable}\spxextra{class in Word}}

\begin{fulllineitems}
\phantomsection\label{\detokenize{index:Word.Syllable}}\pysiglinewithargsret{\sphinxbfcode{\sphinxupquote{class }}\sphinxcode{\sphinxupquote{Word.}}\sphinxbfcode{\sphinxupquote{Syllable}}}{\emph{\DUrole{n}{phonemes}}, \emph{\DUrole{n}{stress}\DUrole{o}{=}\DUrole{default_value}{False}}, \emph{\DUrole{n}{length}\DUrole{o}{=}\DUrole{default_value}{False}}, \emph{\DUrole{n}{tone}\DUrole{o}{=}\DUrole{default_value}{None}}}{}
\sphinxAtStartPar
A class to represent a syllable.

\sphinxAtStartPar
...
\begin{description}
\item[{phonemes}] \leavevmode{[}list{]}
\sphinxAtStartPar
list of the phonemes composing the syllable

\item[{stress}] \leavevmode{[}bool {]}
\sphinxAtStartPar
indicate whether the syllable bears stress or not

\item[{length}] \leavevmode{[}bool {]}
\sphinxAtStartPar
indicate if the vowell in the syllable have more than 2 mora.

\item[{i\_center}] \leavevmode{[}int{]}
\sphinxAtStartPar
index of the phoneme which is at the heart of the syllable

\item[{center: phoneme}] \leavevmode{[}{]}
\sphinxAtStartPar
phoneme which is at the heart of the syllable

\end{description}
\begin{description}
\item[{init\_\_()\_\_}] \leavevmode
\sphinxAtStartPar
constructor that takes as input the list of phonemes in the syllable

\item[{set\_stress(bool) :}] \leavevmode
\sphinxAtStartPar
allow the programm to change the stress of a syllable

\item[{set\_length(bool)}] \leavevmode{[}{]}
\sphinxAtStartPar
allow the programm to change thelength of a syllable

\end{description}
\index{set\_length() (Word.Syllable method)@\spxentry{set\_length()}\spxextra{Word.Syllable method}}

\begin{fulllineitems}
\phantomsection\label{\detokenize{index:Word.Syllable.set_length}}\pysiglinewithargsret{\sphinxbfcode{\sphinxupquote{set\_length}}}{\emph{\DUrole{n}{length}}}{}
\sphinxAtStartPar
Allow the program to change the stress of a syllable
\begin{description}
\item[{stress}] \leavevmode{[}bool{]}
\sphinxAtStartPar
The new value

\end{description}

\sphinxAtStartPar
None.

\end{fulllineitems}

\index{set\_rank\_in\_wd() (Word.Syllable method)@\spxentry{set\_rank\_in\_wd()}\spxextra{Word.Syllable method}}

\begin{fulllineitems}
\phantomsection\label{\detokenize{index:Word.Syllable.set_rank_in_wd}}\pysiglinewithargsret{\sphinxbfcode{\sphinxupquote{set\_rank\_in\_wd}}}{\emph{\DUrole{n}{rk}}}{}
\sphinxAtStartPar
small setter for the rank in word if it changes durong an I change)

\end{fulllineitems}

\index{set\_stress() (Word.Syllable method)@\spxentry{set\_stress()}\spxextra{Word.Syllable method}}

\begin{fulllineitems}
\phantomsection\label{\detokenize{index:Word.Syllable.set_stress}}\pysiglinewithargsret{\sphinxbfcode{\sphinxupquote{set\_stress}}}{\emph{\DUrole{n}{stress}}}{}
\sphinxAtStartPar
Allow the program to change the stress of a syllable
\begin{description}
\item[{stress}] \leavevmode{[}bool{]}
\sphinxAtStartPar
The new value

\end{description}

\sphinxAtStartPar
None.

\end{fulllineitems}


\end{fulllineitems}

\index{Word (class in Word)@\spxentry{Word}\spxextra{class in Word}}

\begin{fulllineitems}
\phantomsection\label{\detokenize{index:Word.Word}}\pysiglinewithargsret{\sphinxbfcode{\sphinxupquote{class }}\sphinxcode{\sphinxupquote{Word.}}\sphinxbfcode{\sphinxupquote{Word}}}{\emph{\DUrole{n}{syls}}}{}
\sphinxAtStartPar
A class to represent a word.

\sphinxAtStartPar
...
\begin{description}
\item[{ipa}] \leavevmode{[}str{]}
\sphinxAtStartPar
phonological transcription of the word using the IPA

\item[{structure}] \leavevmode{[}str {]}
\sphinxAtStartPar
structure of the word (using a CVC format)

\item[{syllables}] \leavevmode{[}list{]}
\sphinxAtStartPar
list of the syllable object the word contains

\item[{phonemes}] \leavevmode{[}list{]}
\sphinxAtStartPar
list of the phonemes the word contains

\item[{phon2syl}] \leavevmode{[}dic{]}
\sphinxAtStartPar
dictionnary mapping the index of a phoneme to the index of the syllable it is in

\end{description}
\begin{description}
\item[{info(additional=""):}] \leavevmode
\sphinxAtStartPar
Prints the person\textquotesingle{}s name and age.

\end{description}
\index{get\_stess\_pattern() (Word.Word method)@\spxentry{get\_stess\_pattern()}\spxextra{Word.Word method}}

\begin{fulllineitems}
\phantomsection\label{\detokenize{index:Word.Word.get_stess_pattern}}\pysiglinewithargsret{\sphinxbfcode{\sphinxupquote{get\_stess\_pattern}}}{}{}
\sphinxAtStartPar
Returns a string representing the stress pattern of the word

\end{fulllineitems}

\index{get\_structure() (Word.Word method)@\spxentry{get\_structure()}\spxextra{Word.Word method}}

\begin{fulllineitems}
\phantomsection\label{\detokenize{index:Word.Word.get_structure}}\pysiglinewithargsret{\sphinxbfcode{\sphinxupquote{get\_structure}}}{}{}
\sphinxAtStartPar
transform a list of syllables into a string representing its structure (in the CVC format)

\end{fulllineitems}


\end{fulllineitems}

\phantomsection\label{\detokenize{index:module-utilitaries}}\index{module@\spxentry{module}!utilitaries@\spxentry{utilitaries}}\index{utilitaries@\spxentry{utilitaries}!module@\spxentry{module}}
\sphinxAtStartPar
Created on Wed May  4 11:30:09 2022

\sphinxAtStartPar
@author: 3b13j

\sphinxAtStartPar
Module containing general functions used to print complex objects or write logs
\index{feature\_indices() (in module utilitaries)@\spxentry{feature\_indices()}\spxextra{in module utilitaries}}

\begin{fulllineitems}
\phantomsection\label{\detokenize{index:utilitaries.feature_indices}}\pysiglinewithargsret{\sphinxcode{\sphinxupquote{utilitaries.}}\sphinxbfcode{\sphinxupquote{feature\_indices}}}{\emph{\DUrole{n}{features}}}{}
\sphinxAtStartPar
Gives the coordinates of the indices of a template
\begin{description}
\item[{features}] \leavevmode{[}TYPE{]}
\sphinxAtStartPar
DESCRIPTION.

\end{description}
\begin{description}
\item[{idx}] \leavevmode{[}TYPE{]}
\sphinxAtStartPar
DESCRIPTION.

\end{description}

\end{fulllineitems}

\index{feature\_match() (in module utilitaries)@\spxentry{feature\_match()}\spxextra{in module utilitaries}}

\begin{fulllineitems}
\phantomsection\label{\detokenize{index:utilitaries.feature_match}}\pysiglinewithargsret{\sphinxcode{\sphinxupquote{utilitaries.}}\sphinxbfcode{\sphinxupquote{feature\_match}}}{\emph{\DUrole{n}{f1}}, \emph{\DUrole{n}{f2}}, \emph{\DUrole{n}{verbose}\DUrole{o}{=}\DUrole{default_value}{False}}}{}
\sphinxAtStartPar
input : two feature list 
returns whether the second is compatible with the first.
It is therefore required to give the most general one as first input.

\end{fulllineitems}

\index{feature\_random\_generator() (in module utilitaries)@\spxentry{feature\_random\_generator()}\spxextra{in module utilitaries}}

\begin{fulllineitems}
\phantomsection\label{\detokenize{index:utilitaries.feature_random_generator}}\pysiglinewithargsret{\sphinxcode{\sphinxupquote{utilitaries.}}\sphinxbfcode{\sphinxupquote{feature\_random\_generator}}}{}{}
\sphinxAtStartPar
Used to generate a feature randomly
\begin{description}
\item[{feature}] \leavevmode{[}feature{]}
\sphinxAtStartPar
randomly generated

\end{description}

\end{fulllineitems}

\index{printd() (in module utilitaries)@\spxentry{printd()}\spxextra{in module utilitaries}}

\begin{fulllineitems}
\phantomsection\label{\detokenize{index:utilitaries.printd}}\pysiglinewithargsret{\sphinxcode{\sphinxupquote{utilitaries.}}\sphinxbfcode{\sphinxupquote{printd}}}{\emph{\DUrole{n}{liste}}}{}
\sphinxAtStartPar
Takes a dictionnary as input and print the first object and the second object, and not the address in memory of the object

\end{fulllineitems}

\index{printl() (in module utilitaries)@\spxentry{printl()}\spxextra{in module utilitaries}}

\begin{fulllineitems}
\phantomsection\label{\detokenize{index:utilitaries.printl}}\pysiglinewithargsret{\sphinxcode{\sphinxupquote{utilitaries.}}\sphinxbfcode{\sphinxupquote{printl}}}{\emph{\DUrole{n}{liste}}}{}
\sphinxAtStartPar
Takes a list of complex objects as input and prints them, one per line

\end{fulllineitems}

\index{tpl\_2\_candidates() (in module utilitaries)@\spxentry{tpl\_2\_candidates()}\spxextra{in module utilitaries}}

\begin{fulllineitems}
\phantomsection\label{\detokenize{index:utilitaries.tpl_2_candidates}}\pysiglinewithargsret{\sphinxcode{\sphinxupquote{utilitaries.}}\sphinxbfcode{\sphinxupquote{tpl\_2\_candidates}}}{\emph{\DUrole{n}{lang}}, \emph{\DUrole{n}{tpl}}, \emph{\DUrole{n}{verbose}\DUrole{o}{=}\DUrole{default_value}{False}}}{}
\sphinxAtStartPar
gives the list of the phonemes of a language that satisfy a conditionned feature template
\begin{description}
\item[{lang}] \leavevmode{[}language{]}
\sphinxAtStartPar
the language we want to extract candidates from

\item[{tpl}] \leavevmode{[}list{]}
\sphinxAtStartPar
feature template that we want to be satisfieds

\end{description}

\sphinxAtStartPar
cands : list

\end{fulllineitems}

\index{vowell() (in module utilitaries)@\spxentry{vowell()}\spxextra{in module utilitaries}}

\begin{fulllineitems}
\phantomsection\label{\detokenize{index:utilitaries.vowell}}\pysiglinewithargsret{\sphinxcode{\sphinxupquote{utilitaries.}}\sphinxbfcode{\sphinxupquote{vowell}}}{\emph{\DUrole{n}{feat}}}{}
\sphinxAtStartPar
checks wether or not the feature given as input encodes a vowell.

\end{fulllineitems}



\renewcommand{\indexname}{Python Module Index}
\begin{sphinxtheindex}
\let\bigletter\sphinxstyleindexlettergroup
\bigletter{c}
\item\relax\sphinxstyleindexentry{Change}\sphinxstyleindexpageref{index:\detokenize{module-Change}}
\item\relax\sphinxstyleindexentry{Condition}\sphinxstyleindexpageref{index:\detokenize{module-Condition}}
\item\relax\sphinxstyleindexentry{conllu\_utilities}\sphinxstyleindexpageref{index:\detokenize{module-conllu_utilities}}
\indexspace
\bigletter{e}
\item\relax\sphinxstyleindexentry{Effect}\sphinxstyleindexpageref{index:\detokenize{module-0}}
\item\relax\sphinxstyleindexentry{encoder\_decoder}\sphinxstyleindexpageref{index:\detokenize{module-encoder_decoder}}
\indexspace
\bigletter{g}
\item\relax\sphinxstyleindexentry{Generator}\sphinxstyleindexpageref{index:\detokenize{module-Generator}}
\indexspace
\bigletter{i}
\item\relax\sphinxstyleindexentry{IPA}\sphinxstyleindexpageref{index:\detokenize{module-IPA}}
\item\relax\sphinxstyleindexentry{IPA\_utils}\sphinxstyleindexpageref{index:\detokenize{module-IPA_utils}}
\indexspace
\bigletter{l}
\item\relax\sphinxstyleindexentry{Language}\sphinxstyleindexpageref{index:\detokenize{module-Language}}
\item\relax\sphinxstyleindexentry{log\_utilities}\sphinxstyleindexpageref{index:\detokenize{module-log_utilities}}
\indexspace
\bigletter{n}
\item\relax\sphinxstyleindexentry{Natural\_class}\sphinxstyleindexpageref{index:\detokenize{module-Natural_class}}
\indexspace
\bigletter{p}
\item\relax\sphinxstyleindexentry{Phoneme}\sphinxstyleindexpageref{index:\detokenize{module-Phoneme}}
\indexspace
\bigletter{r}
\item\relax\sphinxstyleindexentry{rd\_changer}\sphinxstyleindexpageref{index:\detokenize{module-rd_changer}}
\indexspace
\bigletter{s}
\item\relax\sphinxstyleindexentry{Sampling}\sphinxstyleindexpageref{index:\detokenize{module-Sampling}}
\indexspace
\bigletter{t}
\item\relax\sphinxstyleindexentry{Tree}\sphinxstyleindexpageref{index:\detokenize{module-Tree}}
\indexspace
\bigletter{u}
\item\relax\sphinxstyleindexentry{utilitaries}\sphinxstyleindexpageref{index:\detokenize{module-utilitaries}}
\indexspace
\bigletter{w}
\item\relax\sphinxstyleindexentry{Word}\sphinxstyleindexpageref{index:\detokenize{module-Word}}
\end{sphinxtheindex}

\renewcommand{\indexname}{Index}
\printindex
\end{document}